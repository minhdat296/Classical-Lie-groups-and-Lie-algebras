\section{The fundamental theorems of Lie theory}
    \subsection{Lie's three theorems}
        Originally, Sophus Lie tried to encapsulate the relationship between his Lie groups and Lie algebras within three theorems, commonly referred to simply as Lie's First, Second, and Third Theorem. Using the language of categories, however, one is able to combine all three of these results into one singular theorem.
        
        \begin{definition}[Distributions of subspaces] \label{def: distributionsof_subspaces_on_manifolds}
            Suppose that $X$ is a manifold and that $\E \to X$ is a smooth vector bundle. Then, a (smooth) \textbf{distribution of subspaces} of $\E$ is a (smooth) vector sub-bundle $\calF \to X$ of $\E$. When speaking simply of \say{distributions} on $X$, we shall mean smooth vector sub-bundles of te tangent bundle $TM$.
        \end{definition}
        \begin{definition}[Integrability of distributions] \label{def: integrability_of_distributions}
            Let $X$ be a manifold and $\calF \to X$ be a distribution 
            of subspaces thereon. $\calF$ is said to be \textbf{integrable} if its fibres $\calF_x$ over points $x \in X$ are Lie algebras with respect to the natural bracket defined via composition (i.e. for all sections $\del_1, \del_2 \in \calF_x$, we define $[\del_1, \del_2] := \del_1 \circ \del_2 - \del_2 \circ \del_1$). 
        \end{definition}
        \begin{example}
            \noindent
            \begin{itemize}
                \item Distributions of subspaces of the tangent bundle over a Lie group are always integrable.
                \item Let $v := \del_x$ and $w := x\del_y + \del_z$ and consider the distribution $\calF := \span_{\R}\{v, w\}$ over $\R^3$. Clearly $[v, w] \not \in \calF$, and therefore $\calF$ is not an integrable distribution over $\R^3$.
            \end{itemize}
        \end{example}
        \begin{definition}[Integral submanifolds] \label{def: integral_submanifolds}
            
        \end{definition}
        \begin{lemma}[An integrability criterion for distributions] \label{lemma: an_integrability_criterion_for_distributions}
            
        \end{lemma}
            \begin{proof}
                
            \end{proof}
        
        \begin{lemma}[Embeddings of finite-dimensional Lie algebras into matrix algebras] \label{lemma: embedding_finite_dimensional_lie_algebras_into_matrix_algebras}
            Any finite-dimensional Lie algebra over a given field $k$ of characteristic $0$ is a Lie subalgebra of $\frakgl_n(k)$ for some $n \geq 1$.
        \end{lemma}
            \begin{proof}
                Recall that the universal enveloping algebras of Lie $k$-algebras arise via a (strict) monoidal functor:
                    $$U: k\-\Lie\Alg \to k\-\bi\Alg$$
                    $$\g \mapsto U(\g)$$
                where on the left-hand side, the monoidal structure is given by binary direct sums of Lie algebras, with the monoidal unit being the zero Lie algebra, i.e. we have $U(\g \oplus \h) \cong U(\g) \tensor_k U(\h)$ for all Lie $k$-algebras $\g$ and $\h$, and that for each Lie $k$-algebra $\g$ the corresponding universal enveloping algebra $U(\g)$ is a $k$-bialgebra\footnote{Always assumed to be (co)unital and (co)associative.}. Recall also that there is a monoidal equivalence:
                    $$\Rep_k(\g) \cong U(\g)\bimod$$
                between the category of $k$-linear representations of $\g$ and that of $U(\g)$-bimodules.
            \end{proof}
        
        \begin{theorem}[The Fundamental Theorem of Lie Theory] \label{theorem: fundamental_theorem_of_lie_theory}
            Suppose that $k$ is either $\R$ or $\bbC$, equipped with their natural archimedean topologies and consider the following functor from the category of Lie groups over $k$ to that of Lie $k$-algebras:
                $$\frakLie: \Lie\Grp_{/k} \to k\-\Lie\Alg$$
                $$G \mapsto T_e G$$
            which assigns to each Lie group $G$ over $k$ its tangent space at the identity $T_eG$ (which shall be understood to come equipped with its natural Lie algebra structure). 
                \begin{enumerate}
                    \item The functor $\frakLie$ preserves and reflects monomorphisms.
                    \item The functor $\frakLie$ is fully faithful. 
                    \item In fact, it gives an equivalence between the category of simply connected Lie groups over $k$ and that of finite-dimensional Lie $k$-algebras.
                \end{enumerate}
        \end{theorem}
            \begin{proof}
                \noindent
                \begin{enumerate}
                    \item 
                    \item 
                    \item 
                \end{enumerate}
            \end{proof}
        
        \begin{proposition}
            Suppose that $k$ is either $\R$ or $\bbC$, equipped with the usual archimedean topology. Any connected Lie group $G$ over $k$ thus has the form $G_0/\Gamma$ where $G_0$ is a simply connected Lie group over $k$, and $\Gamma$ is a discrete central\footnote{Hence normal.} subgroup thereof.
        \end{proposition}
            \begin{proof}
                
            \end{proof}
    
    \subsection{Basics on representations of Lie algebras}
    
    \subsection{The Baker-Campbell-Hausdorff Formula}